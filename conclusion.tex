\section{Conclusions \& Future Work}
\label{conclusion}

The WORG method provides a deployment schedule optimizer that converges 
both closely enough and fast enough to be used inside
of a nuclear fuel cycle simulator. The algorithm can consistently obtain
tolerances of half-a-percent to a percent (1 GWe distances for over 200 GWe
deployable) for the once-through fuel cycle featured here within only five to 
ten simulations. Such optimization problems are made
more challenging due to the integral nature of facility deployment and
that any demand curve may be requested.

WORG works by setting up a Gaussian process to model the production
as a function of time and the deployment schedule. This model may then
be evaluated orders of magnitude faster than running a full simulation, enabling
the search over many potential deployment schedules. The quality of these
possible schedules is evaluated based on the dynamic time warping distance
to the demand curve. The lowest distance curve is then evaluated in a
full fuel cycle simulation. The production curve that is computed by the 
simulator in turn goes on to update the Gaussian process model and the
cycle repeats until the limiting conditions are met.

However, choosing the deployment schedules to estimate with the Gaussian
process may be performed in a number of ways. A blind approach would
simply be to choose such schedules randomly from a univariate. However, 
the WORG method has more information available to it that helps drive 
down the number of loop iterations. The first method discussed remains 
stochastic but uses the inverse DTW distances of the GP model to 
weight the deployment options, falling back to a Poisson distribution as 
necessary. This second method minimizes the model distance for each point 
in time from start to end, iteratively building up a solution. Finally, 
another estimation strategy tries both previous options and chooses the 
best result, forcing the stochastic method two of every four iterations 
to avoid deterministic loops.  It is this last all-of-the-above method 
that is seen to converge the fastest and to the lowest distance in most 
cases.

It is important to note that the WORG algorithm is applicable to any 
demand curve type and fuel cycle facility type. It is not restricted to 
reactors and power.  Enrichment and separative work units, reprocessing
and separations capacity, and deep geologic repositories and their
space could be deployed via the WORG method for any applicable demand 
curve.  Reactors were chosen for study here as the representative keystone 
example.

The next major step for this work is to actually employ the WORG method in 
a fuel cycle simulator.  However, to the best knowledge of the 
author, no existing simulator is capable of spawning forks of itself 
during run time, rejoining the processes, and evaluating the results of the 
child simulations in the parent simulation. Concisely, while many simulators 
are 
`dynamic' in the fuel cycle sense, none are `dynamic' in the programming
language sense. This latter usage of the term is what is required to 
take advantage of any sophisticated \emph{in situ} deployment optimizer.
The Cyclus fuel cycle simulator looks most promising as a platform
for such work to be undertaken. However, many technical roadblocks 
on the software side remain, even for Cyclus.

Furthermore, adding \emph{in situ} capability also adds the additional 
degree of freedom of how often to run the deployment schedule optimizer.
Running WORG each and every
time step seems excessive \emph{a priori}. Is every year, five years,
or ten years sufficient? How does this degree of freedom balance with the
time horizon $T$ specificed in the optimizer? These questions remain unanswered, even 
in a heuristic sense, and thus the frequency of optimization will be a key 
parameter in a future \emph{in situ} study.
