\section{Conclusions \& Future Work}
\label{conclusion}

The WORG method provides a deployment schedule optimizer that coverges 
both closely enough and fast enough for an \emph{in situ} optimizer inside
of a nuclear fuel cycle simulator. The alogorithm can consistenly obtain
tolerances of half-a-percent to a percent (1 GWe distances for over 200 GWe
deployable) for the once-through fuel cyelc featured here in only five to 
ten simulations. Such optimization problems are made
more challenging due to the integral nature of facility deployment and
that any demand curve may be requested.

WORG works by setting up a Gaussian process to determine model the production
as a function of time and the deployment schedule. This model may then
be evalued orders of magnitude faster than running a full simulation, enabling
the search over many potential deployment schedules. The quality of these
possible schedules is evalued based on the dynamic time warping distance
to the demand curve. The lowest disnatce curve is then evaluated in a
full fuel cycle simulation. The production curve that is computed by the 
simululator in turn goes on to update the Gaussian process model and the
cycle repeats until the limiting conditions are met.

However, choosing the deployment schedules to esitimate with the Gaussian
process may be performed in any number of way. A blind approach would
simply be to choose such schedules randomly from a univariate. However, 
the WORG method has more information available to it that helps drive 
down the number of loop iterations. This first method discussed remains 
stochasitic but uses the inverse DTW distances of the GP model to 
weight the deployment options, falling back to a Poisson distribution as 
necessary. This second method minimizes the model distnace for each point 
in time from start to end, iteratively building up a solution. Finally, 
another estimation strategy tries both previous options and chooses the 
best result, forcing the stochastic method two of every four iterations 
toi avoid deterministic loops.  It is this last all-of-the-above method 
that is seens to coverge the fastest and to the lowest distance in most 
cases.

It is important to note that the WORG algorithm is applicable to any 
demand curve type and fuel cycle facility type. It is not restricted to 
reactors and power.  Enrichment and seperative work units, reprocessing
and separeations capcity, and deep geologic repositories and their
space could be deployed via the WORG method for any applicable demand 
curve.  Reactors were chosen for study here as the representive keystone 
example.

The next major step for this work is to actually employ the WORG method in 
a fuel cycle simulator.  However, to the best of the knowlegde of the 
author, no existing simulator is capable of spawing forks of itself 
during run time, rejoining the proceess, and evaluating the results of the 
child ssimulation in the parent. Concisely, while many simulators are 
`dynamic' in the fuel cycle sense, none are `dynamic' in the programming
language sense. This latter usage of the term is what is required to 
take advantage of any sophisticated \emph{in situ} deployement optimizing.
The Cyclus fuel cycle simulator looks most promising as a platform
for such work to be undertaken. However, many technical roadblock remain.

Furthermore, adding \emph{in situ} capability also adding the additional 
degree of freedom for how often to run the deployment schedule optimizer.
Every time step seems excessive \emph{a priori}. Is every year, five years,
ten years good enough? The fequency of optimization will be a key 
parameter in any future study.

In summary, the WORG algorithm provides fast and sufficiently close 
deployment schedule optimization for any demand curve.  Many kinds of 
facilities may be deployed to meet the same demand.  The method 
limits the number of full fuel cycle simulations that must be performed by
managing stochastic and inner product estimation methods. WORG
evaluates potential deployment schedules via a dynamic time warping 
comparison of Gaussian process models. The work here shows that this 
method is a prime candicate for future \emph{in situ} deployment 
optimization. 


