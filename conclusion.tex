\section{Conclusions \& Future Work}
\label{conclusion}

The WORG method provides a deployment schedule optimizer that coverges 
both closely enough and fast enough for an \emph{in situ} optimizer inside
of a nuclear fuel cycle simulator. The alogorithm can consistenly obtain
tolerances of half-a-percent to a percent (1 GWe distances for over 200 GWe
deployable) for the once-through fuel cyelc featured here in only five to 
ten simulations. Such optimization problems are made
more challenging due to the integral nature of facility deployment and
that any demand curve may be requested.

WORG works by setting up a Gaussian process to determine model the production
as a function of time and the deployment schedule. This model may then
be evalued orders of magnitude faster than running a full simulation, enabling
the search over many potential deployment schedules. The quality of these
possible schedules is evalued based on the dynamic time warping distance
to the demand curve. The lowest disnatce curve is then evaluated in a
full fuel cycle simulation. The production curve that is computed by the 
simululator in turn goes on to update the Gaussian process model and the
cycle repeats until the limiting conditions are met.

However, choosing the deployment schedules to esitimate with the Gaussian
process may be performed in any number of way. A blind approach would
simply be to choose such schedules randomly from a univariate. However, 
the WORG method has more information available to it that helps drive 
down the number of loop iterations. This first method discussed remains 
stochasitic but uses the inverse DTW distances of the GP model to 
weight the deployment options, falling back to a Poisson distribution as 
necessary. This second method minimizes the model distnace for each point 
in time from start to end, iteratively building up a solution. Finally, 
another estimation strategy tries both previous options and chooses the 
best result, forcing the stochastic method two of every four iterations 
toi avoid deterministic loops.  It is this last all-of-the-above method 
that is seens to coverge the fastest and to the lowest distance in most 
cases.

It is important to note that the WORG algorithm is applicable to any 
demand curve type and fuel cycle facility type. It is not restricted to 
reactors and power.  Enrichment and seperative work units, reprocessing
and separeations capcity, and deep geologic repositories and their
space could be deployed via the WORG method for any applicable demand 
curve.  Reactors were chosen as the representive keystone example.

