\section{Performance \& Results}
\label{results}

To demonstrate the three variant WORG methods, an unconstrainted once 
through fuel cycle is modeled with the Cyclus simulator 
\cite{DBLP:journals/corr/HuffGCFMOSSW15}. In such a scenario, uranium
mining, enrichment, fuel fabrication, and storage all have effectively 
infinite capacities. The only meaningful constraints on the system are
how many light-water reactors (LWR) are built.

The base simulation begins with 100 reactors in 2016 that each produce
1 GWe, have an 18 month batch legnth with a one month reload time.
The initial fleet of LWRs retires evenly over the 40 years from 2016 to 
2056. All new reactors have a 60 year life time.  The simulation itself 
follows 50 years from 2016 to 2066. \emph{In situ} time horizons are 
expected to be much lower, on the order of 5, 10, or 20 years.

The study here compares how WORG performs from 0\% (steady state), 1\%, 
and 2\% growth curves from an initial 90 GWe target. These are examined
using the three estimation variants.  Calling $r$ the growth rate as a 
fraction, the demand curve is thus,
\begin{equation}
\label{f-rate}
f(t) = 90 (1 + r)^t
\end{equation}
Moreover, the upper bound for the number of deployable facilities at 
each time is set to be the ceiling of four times the total growth. 
That is, assuming four facilities at most could be deployed in the first
year, increase the upper bound along with the growth rate.  This yields
the following expression for $N$.
\begin{equation}
\label{n-rate}
N(t) = \left\lceil 4 (1 + r)^t\right\rceil
\end{equation}
The lower bound for the number of deployed reactor is simply set to the 
zero vector, $M = \mathbf{0}$.