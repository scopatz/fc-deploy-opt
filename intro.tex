\section{Introduction}
\label{intro}

With the recent advent of agent-based nuclear fuel cycle simulators, such as 
Cyclus \cite{DBLP:journals/corr/HuffGCFMOSSW15,cyclus_v1_0}, there comes the 
possibility to make \emph{in situ}, dynamic facility deployment decisions.
This  would more fully model real-world fuel cycles where institutions 
(such as utility companies)
predict future demand and choose their future deployment schedules 
appropriately. However, one of the major challenges to making \emph{in situ}
deployment decisions is the speed at which ``good enough'' decisions can 
be made. This paper proposes three related deployment-specific optimization 
algorithms that can be used for any demand curve and facility type.

The demands of a fuel cycle scenario can often be simply stated, e.g. 
1\% growth in power production [GWe]. Picking a deployment schedule for a 
certain kind of facility (e.g. reactors) can thus be seen as an optimization 
problem of how well the deployment schedule meets the demand. Here, the 
dynamic time warping (DTW) \cite{muller} distance is minimized 
between the demand curve and the regression of a Gaussian Process model (GP) 
\cite{rasmussen2006gaussian} of prior simulations. This minimization produces
a guess for a deployment schedule which is subsequently tested using 
the actual simulator. This process is repeated until an optimal deployment
schedule for the given demand is found.

Importantly, by using the Gaussian process surrogates, the number of 
simulation realizations that must be executed as part of the optimization may 
be reduced to only a handful. Furthermore, it is at least two 
orders-of-magnitude faster to test the model than it is to run a single
low-fidelity fuel cycle simulation. Because of the relative computational 
cheapness, it 
is suitable to be used inside of a fuel cycle simulation. Traditional
\emph{ex situ} optimizers may be able to find more precise solutions but at a
computational cost beyond the scope and need of an \emph{in situ} use case.

Every iteration of the warp optimization of regressed Gaussian processes (WORG) 
method described here has two phases. The first is an estimation phase where 
the Gaussian process model is built and evaluated. The second takes the 
deployment schedule from the estimation phase and runs it in a fuel cycle 
simulator. The results of the simulator of the $s$-th iteration are then 
used to inform the model on the $(s+1)$-th iteration. 

Inside of each estimation phase there are three possible strategies for 
choosing the next deployment schedule.  The first is to sample of the 
space of all possible deployment strategies stochastically and then take the 
best guess.  The second is to search through the inner product of all choices,
picking the best option for each deployment parameter. The third strategy
is to perform the two previous strategies and determine which one has picked
the better guess.

Nuclear fuel cycle demand curve optimization faces many challenges. 
Foremost among these is that even though the demand curve is specified on 
the range of the real numbers, the optimization parameters are fundamentally 
integral in nature. For a discrete time simulator, deployments can only 
be issued in multiples of the size of the time step 
\cite{kelton2000simulation}. Furthermore, 
it is not possible to deploy only part of a facility; the facility is either 
deployed or it is not. While it may be possible to deploy a facility and 
only run it at partial capacity, most fuel cycle models do not support such
a feature for
keystone facilities.  For example, it is unlikely that a utility would build 
a new reactor only to run it at 50\% power. Thus, deployment is an integer 
programming problem, as opposed to its easier linear programming cousin
\cite{vanderbei2001linear}.

As an integer programming problem, the option space is combinatorially 
large. Assuming a 50 year deployment schedule where no more than 3 facilities 
are allowed to be deployed each time step, there are more than $10^{30}$ 
combinations. If every simulation took a very generous 1 sec, simulating 
each option would still take $\approx 3\times10^{12}$ times the current age
of the universe.

Moreover because all of the parameters are integral, there is not a 
meaningful formulation of a continuous Jacobian. Derivative-free optimizers are 
required. Methods such as particle swarm \cite{kennedy2010particle}, 
pswarm \cite{vaz2009pswarm}, and the 
simplex method \cite{vanderbei2001linear} all could work.  However, typical 
implementations require
more evaluations of the objective function (i.e. fuel cycle simulations)
than are within an \emph{in situ} budget. 

Even the usual case of 
Gaussian process optimization (sometimes known as kriging) 
\cite{osborne2009gaussian,simpson2001kriging} will still 
require too many full realizations in order to form an accurate model.
WORG, on the other hand, uses the dynamic time warping distance as a 
measure of how two time series differ. This is because the DTW distance is 
more separative than the typical
$L_1$ norm. Such additional separation drives the estimation phase to make 
better choices
sooner. This in turn helps the overall algorithm converge on a reasonable 
deployment schedule sooner. 
The stochastic strategy for WORG additionally utilizes Gaussian processes to 
weight the choice of parameters. This guides the guesses for the deployment
schedules such that fewer guesses are needed while simultaneously 
not forbidding  
any option.  So while WORG relies on Gaussian processes, it does so in a way
that is distinct from normal kriging. WORG
takes advantage of the \emph{a priori} knowledge that a deployment 
schedule is requested to meet a demand curve. 

The structure of the WORG algorithm is detailed in \S\ref{method}. 
The different strategies for selecting a best guess estimate of the 
deployment schedule are then discussed in \S\ref{selecting}. Performance
and results of the method for a sample once-through fuel cycle scenario 
are presented in \S\ref{results}. Finally, \S\ref{conclusion} summarizes
WORG and lists opportunities for future work.
