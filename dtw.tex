\subsection{Dynamic Time Warping}
\label{dtw}

The question of how to take the difference between the demand curve and 
the production curve is an important one. The na\"ive option is to simply 
take the $L_1$ norm of the difference between these two time sereies, as 
seen in Equation \ref{delta-l1}.  However, since $g(t, \Theta)$ for 
a simulation is expensive compute, any operation that can meaningfully 
exacerbate the difference betweeen time series helps drive down the number 
of optimization iterations.

Dynamic time warping is just such a mechanism. It computes 
a distance between any two time series which compounds the separartion 
between the two. Additionally, the time series are not required to be of the 
same length, though for optimization purposes there is no reason for them 
not to be. DTW gives a measure of amount that one time series would need to 
be warped to become the other time series. It is, therefore, a holistic  
measure that operates over the complete time series. Dynamic time warping
is more fully covered in \cite{muller}.  However, an 
optimization-relevant introduction is given here.

For the time series $f$ and $g$, there are three parts to dynamic time 
warping. The first is the distance $d$, which will be minimized. The second 
is a cost matrix $C$ that helps compute $d$ by indicating how far a point 
on $f$ is from another point on $g$. Thirdly, the warp path $u$ is the 
minimal cost curve through the $C$ matrix from the fist point in time to 
the last. The DTW distance can thus be interpreted as the 
total cost of travelling the warp path.

The first step in computing a dynamic time warp distance is to 
assemble the cost matrix. Say that the demand time series $f$ has 
length $A$ indexed by $a$ and the production time series $g$ has 
length $B$ indexed by $b$. For the optimization problem here, $A$ and $B$
are in practice both equal to $T$.  However, it is useful to have $a$ and 
$b$ index the two time series separately. Now denote an $A\times B$ matrix 
$\Delta L$ as the $L_1$ norm of the difference between $f$ and $g$:
\begin{equation}
\label{delta-l1}
\Delta L_{a,b} = \left|f(a) - g(b, \Theta)\right|_1
\end{equation}
The cost matrix $C$ may now be defined as the $A\times B$ sized matrix 
which follows the recursion relations seen in Equation \ref{cost-matrix}.
\begin{equation}
\label{cost-matrix}
\begin{split}
C_{1,1} & = \Delta L_{1,1}\\
C_{1,b+1} & = \Delta L_{1,b} + C_{1,b}\\
C_{a+1,1} & = \Delta L_{a,1} + C_{a,1}\\
C_{a+1,b+1} & = \Delta L_{a,b} + \min\left[C_{a,b}, C_{a+1,b}, C_{a,b+1}\right]
\end{split}
\end{equation}
The boundary conditions above are the same as setting an infinite cost to 
any $a \le 0$ or $b \le 0$. The cost matrix $C$ has the same units as the 
demand curve. However, the scale of $C$ is (except for the fiducial case) 
larger than the demand. This is because the cost matrix compounds the 
minimum value of previous entries. 

Knowing a cost matrix, the warp path can be computed by traversing the 
matrix backwards from the $(A, B)$ corner to the $(1, 1)$ corner.
If the length of the warp is $I$ indexed by $i$, the warp path itself 
can be thought of as a sequence of coordinate points $u_i$. For a given 
point $u_i$ in the warp path, the previous point $u_{i-1}$ found by 
picking the minimum cost point among the locations one column over $(a,b-1)$, 
one row over $(a-1,b)$, and one previous diagonal element to $(a-1,b-1)$. 
Equation \ref{warp-path} expresses this mathematically.
\begin{equation}
\label{warp-path}
u_{i-1} = \argmin\left[C_{a-1,b-1}, C_{a-1,b}, C_{a,b-1}\right]
\end{equation}
The maximum possible length of $u$ is thus $\max(I) = A + B$.
The minimum possible length, though, is $\min(I) = \sqrt{A^2 + B^2}$. 

The dynamic time warping distance distance $d$ can now be stated as the 
cost of the final entry of the warp path normalized by the maximum possible
length of the warp path.  
\begin{equation}
\label{d-calc-ab}
d(f, g) = \frac{C_{A,B}}{A + B}
\end{equation}
However, because the demand curve and the production curve that is predicted
or calculated are typically defined on the same time grid, $d$ can be further
reduced to the following:
\begin{equation}
\label{d-calc}
d(f, g) = \frac{C_{T,T}}{2T}
\end{equation}
Therefore, $d$ has the same units as the demand curve, production curve, 
and cost matrix.

As an example ....

\begin{figure}[htb]
\centering
%\includegraphics[width=0.9\textwidth]{}
\caption{Heat map of the cost matrix between ...
The warp path is superimposed as the white curve on top of the cost matrix.}
\label{}
\end{figure}

Figure \ref{} displays an example cost matrix 
as a heat map for the DTW between ...
Additionally, the warp path between these two is shown as the white curve
on top of the heat map. Note that while $u$ is monotonic along both time axes, the
path it takes minimizes the cost matrix at every step. Higher cost regions
have the effect of pushing the warp path along one axis or another. The 
distance between these two curves $d(f, \vec{0})$ is computed 
to be XYZ GWe.

In summary, dynamic time warping yields a meaningful mechanism for ...
