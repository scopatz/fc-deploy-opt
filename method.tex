\section{The WORG Method}
\label{method}

In order to describe the WORG method, first it is is useful to define  
notation for demand curves and their parameterization.  Call $t$ the time
[years] up to some maximal time $T$ (e.g. 50 years) over which the time 
demand curve is know.  Then call $f(t)$ the demand curve, in the natural 
units of the facility type ([GWe] for reactors).
$f(t)$ may be any function that is desired, including non-differential 
functions. For example, though, the demand curve for a 1\% growth rate 
starting at 90 [GWe] has the following form:
\begin{equation}
\label{f-1}
f(t) = 90\times 1.01^t
\end{equation}
Additionally, call $\Theta$ the deployment schedule for the facility.
$\Theta$ is a sequence of $P$ parameters, indexed by $p$, as seen in 
Equation \ref{Theta}.
\begin{equation}
\label{Theta}
\Theta = \left\{\theta_1, \theta_2, \ldots, \theta_P\right\}
\end{equation}
Each $\theta_p$ represents that number of facilities to deploy on its
time step. In simple cases where there is only one type of facility
to deploy $P == T$.  However, when the deployment schedules of multiple 
facility types are needed to meet the same demand curve, $P > T$.  The usual
example is for transition scenarios which necessarily require multiple kinds
of reactors.

Now denote $M$ as the sequence for the minimum number of facilities deployable
for the $p$-th deployment. Then call $N$ the sequence of the maximum number
of facilcities deployable. The deployment parameters are thus each defined
on the range $\theta_p \in [M_p, N_p]$. Furthermore, because only whole
numbers of facilities may be deployed $\theta_p \in \N$.  It is also typical, 
but not required for $M = \mathbf{\vec{0}}$, which is also the lower bound
for all possible $\theta_p$ since faciliteis may not be retired by the 
deployment schedule.

From here, call $g(t, \Theta)$ the production as a function of time for a
given deployment facility. This has the same units as the demand curve.
Thus for power demand and reactor deployments $g$ is in [GWe]. The 
optimization problem can now be posed as an attempt to find a $\Theta$
that minimizes the difference between $f$ and $g$.

